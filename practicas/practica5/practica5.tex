\documentclass[10pt]{article}
\usepackage{amsmath} 

\begin{document}

\textbf{Problema 1: Funciones homog\'eneas}
\\

Sea una funci\'on homog\'enea generalizada

\begin{equation}
\lambda f(x, y) = f(\lambda^a x, \lambda^b y).
\end{equation}

Muestre que la transformada de Legendre 

\begin{equation}
g(x, u(x,y)) = f(x, y) - y u(x, y),
\end{equation}

donde 

\begin{equation}
u(x, y) = \left(\dfrac{\partial f}{\partial y} \right)_x,
\end{equation}

es tambi\'en una funci\'on homog\'enea generalizada.\\

\textbf{Soluci\'on:}
\\

Observemos primero que la derivada de una funci\'on homog\'enea generalizada es tambi\'en funci\'on homog\'enea generalizada:

\begin{align*}
\lambda \dfrac{\partial f(x, y)}{\partial y} &= \dfrac{\partial f(\lambda^a x, \lambda^b y)}{y} \\
\lambda u(x, y) &= \lambda^b \dfrac{\partial f(\lambda^a x, \lambda^b y)}{\partial (\lambda^b y)} \\
\lambda^{1-b} u(x, y) &= u(\lambda^a x, \lambda^b y)
\end{align*}

Ahora bien, partiendo de la definici\'on de $g$,

\begin{align*}
f(\lambda^a x, \lambda^b y) &= g(\lambda^a x, u(\lambda^a x,\lambda^b y)) + \lambda^b y u(\lambda^a x, \lambda^b y) \\
\lambda f(x, y) &= g(\lambda^a x, \lambda^{1-b} u(x, y)) + \lambda^b y \lambda^{1-b} u(x, y) \\
\lambda \left( f(x, y) - y u(x, y) \right) &= g(\lambda^a x, \lambda^{1-b} u(x, y) \\
\lambda^{1-b} g(x, u) &= g(\lambda^a x, \lambda^{1-b} u)
\end{align*}

\pagebreak

\textbf{Problema 2: Teor\'ias de escala}\\

Usando la construcci\'on de bloques de Kadanoff muestre que la funci\'on de correlaci\'on de pares para un ferromagneto puede expresarse como 

\begin{equation}
C(r, t, B) = t^{\nu(d-2+\eta)} F\left( \dfrac{r}{t^{-\nu}}, \dfrac{B}{t^{\Delta}} \right),
\end{equation}

donde $t = T/T_c - 1$ y $\Delta = \beta \delta$.\\


\textbf{Soluci\'on:}\\

\begin{equation}
b^{2(d-y)}C(r, t, B) = C(r/b, t b^x, B b^y)
\end{equation}

En particular, para $t b^x = 1$ (o bien, $b = t^{-1/x}$), 

\begin{align*}
t^{-2(d-y)/x} C(r, t, B) = C\left( \dfrac{r}{t^{1/x}}, \dfrac{B}{t^{y/x}} \right).
\end{align*}

Usando las identidades $\nu = 1/x$, $\Delta = y/x$ y $2(d-y) = d - 2 - \eta$, se deduce que

\begin{equation}
C(r, t, B) = t^{\nu(d-2+\eta)} F\left( \dfrac{r}{t^{-\nu}}, \dfrac{B}{t^{\Delta}} \right),
\end{equation}

\end{document}