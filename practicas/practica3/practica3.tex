\documentclass[10pt]{article}
\usepackage{amsmath} 

\begin{document}

\textbf{Problema 1: Modelo de Blume-Capel - Aproximaci\'on variacional}
\\

\begin{equation}
H = -J \sum_{\langle i,j\rangle} s_i s_j + D \sum_i s_i^2
\end{equation}

\begin{equation}
H_0 = -\eta \sum_i s_i + D \sum_i s_i^2
\end{equation}

Desigualdad de Bogoliuvob-Peierls:

\begin{equation}
f \leq f_{\rho} = f_0 + \dfrac{1}{N} \langle H - H_0 \rangle_0.
\end{equation}

Funci\'on de partici\'on para el hamiltoniano de prueba:

\begin{align}
\mathcal{Z}_0 &= \sum_{\lbrace s_i\rbrace} e^{-\beta H_0} \nonumber \\
&= \sum_{\lbrace s_i\rbrace} \exp \left[ \beta \eta \sum_i s_i - \beta D \sum_i s_i^2 \right] \nonumber \\
&= \sum_{\lbrace s_i\rbrace} \prod_i  \exp \left[ \beta \eta s_i - \beta D  s_i^2 \right] \nonumber \\
&= \prod_i \sum_{s_i=0,\pm1} \exp \left[ \beta \eta s_i - \beta D  s_i^2 \right] \nonumber \\
&= \left[1 + e^{-\beta \eta -\beta D} + e^{\beta \eta -\beta D} \right]^N \nonumber \\
&= \left[1 + 2e^{-\beta D} \cosh\left(\beta \eta\right) \right]^N, \nonumber \\
&= \mathcal{Z}_{01}^N,
\end{align}

donde 

\begin{equation}
\mathcal{Z}_{01} = 1 + 2e^{-\beta D} \cosh\left(\beta \eta\right).
\end{equation}

Energ\'ia libre:

\begin{align}
f_0 &= -\dfrac{1}{\beta N} \ln \mathcal{Z}_0 \nonumber \\
&= -\dfrac{1}{\beta} \ln \left[ 1 + 2e^{-\beta D} \cosh\left(\beta \eta\right) \right]
\end{align}

Magnetizaci\'on:

\begin{align} \label{eq:BC_m0}
m_0 &= \langle s_i \rangle_0 \nonumber \\
&= \dfrac{1}{\mathcal{Z}_{01}} \sum_{s_i=0,\pm1} s_i e^{\beta \eta s_i - \beta D  s_i^2} \nonumber \\
&= \dfrac{1}{\mathcal{Z}_{01}} \left[-e^{-\beta \eta -\beta D} + e^{\beta \eta -\beta D} \right] \nonumber \\
&= \dfrac{2e^{-\beta \eta}\sinh\left( \beta \eta \right)}{1+2e^{-\beta \eta}\cosh\left( \beta \eta \right)}
\end{align}

Segundo momento:

\begin{align}
\langle s_i^2 \rangle_0 &= \dfrac{1}{\mathcal{Z}_{01}} \sum_{s_i=0,\pm1} s_i^2 e^{\beta \eta s_i - \beta D  s_i^2} \nonumber \\
&= \dfrac{1}{\mathcal{Z}_{01}} \left[e^{-\beta \eta -\beta D} + e^{\beta \eta -\beta D} \right] \nonumber \\
&= \dfrac{2e^{-\beta \eta}\cosh\left( \beta \eta \right)}{1+2e^{-\beta \eta}\cosh\left( \beta \eta \right)}
\end{align}

Valores medios de los hamiltonianos respecto del hamiltoniano de prueba:

\begin{align}
\langle H_0 \rangle_0 &= -\eta \sum_i \langle_0 s_i \rangle + D \sum_i \langle s_i^2 \rangle_0 \nonumber \\
&= -\eta N m_0 + D N \langle s_i^2 \rangle_0 \nonumber
\end{align}

\begin{align}
\langle H \rangle_0 &= -J \sum_{\langle i,j\rangle} \langle s_i s_j \rangle_0 + D \sum_i \langle s_i^2 \rangle_0 \nonumber \\
&= -J \sum_{\langle i,j\rangle} \langle s_i \rangle_0 \langle s_j \rangle_0 + D \sum_i \langle s_i^2 \rangle_0 \nonumber \\
&= -J \dfrac{Nz}{2} m_0^2 + D N \langle s_i^2 \rangle_0 \nonumber
\end{align}

Proponemos la funci\'on variacional 

\begin{align} \label{eq:BC_Phi}
\Phi(\eta) &= f_0 + \dfrac{1}{N} \langle H - H_0 \rangle_0 \nonumber \\
&= -\dfrac{1}{\beta} \ln \left[ 1 + 2e^{-\beta D} \cosh\left(\beta \eta\right) \right] - J \dfrac{z}{2} m_0^2 + \eta m_0
\end{align}

Derivamos con respecto a $\eta$ e igualamos a cero para hallar el m\'inimo

\begin{align}
\dfrac{\partial \Phi}{\partial \eta} &= -  \dfrac{2e^{-\beta \eta}\sinh\left( \beta \eta \right)}{1+2e^{-\beta \eta}\cosh\left( \beta \eta \right)} - Jzm_0 \dfrac{\partial m_0}{\partial \eta} + m_0 + \eta \dfrac{\partial m_0}{\partial \eta} \nonumber \\
&= \left(\eta - Jzm_0\right) \dfrac{\partial m_0}{\partial \eta},
\end{align}

donde usamos la igualdad \ref{eq:BC_m0}.

La expresi\'on anterior implica que la soluci\'on al problema variacional es 

\begin{equation}
\eta = Jzm_0.
\end{equation}

Reemplazando en \ref{eq:BC_Phi}, tenemos

\begin{equation}\label{eq:Phi_var}
\Phi(\eta) = -\dfrac{1}{\beta} \ln \left[ 1 + 2e^{-\beta D} \cosh\left(\beta \eta\right) \right]  + \dfrac{1}{2Jz} \eta^2
\end{equation}

Realizamos un desarrollo de Taylor de orden 4 de la expresi\'on anterior, teniendo en cuenta la siguiente relaci\'on:

\begin{align*}
\ln \left[ 1+ 2 c \cosh(x) \right] &= \ln(2c+1) + \dfrac{c}{2c+1} x^2 + \\
&+ \dfrac{c(1-4c)}{12(2c+1)^2} x^4 + \dfrac{c(64c^2-26c+1)}{360(2c+1)^3} x^6.
\end{align*}

Reemplazando en \ref{eq:Phi_var} y definiendo $c = e^{-\beta D}$,

\begin{align*}
\Phi(\eta)  &= -\dfrac{1}{\beta}\ln(2c+1) +\left( \dfrac{1}{2Jz} - \beta \dfrac{c}{2c+1}\right) \eta^2 + \\
&+ \beta^4\dfrac{c(1-4c)}{12(2c+1)^2} \eta^4 + \beta^6\dfrac{c(64c^2-26c+1)}{360(2c+1)^3} \eta^6.
\end{align*}

El punto tricr\'itico se obtiene igualando a cero los coeficientes de $\eta^2$ y $\eta^4$. A partir del t\'ermino cuadr\'atico se obtiene

\begin{align*}
\dfrac{1}{2Jz} &= \beta \dfrac{c}{2c+1} \\
2 + c^{-1} &= 2\beta J z \\
\beta D &= \ln 2 + \ln(\beta J z - 1)
\end{align*}

mientras que del t\'ermino cu\'artico,

\begin{align*}
c &= \dfrac{1}{4} \\
\beta D &= 2 \ln 2
\end{align*}

Despejando,

\begin{align*}
D &= \dfrac{2Jz}{3}\ln 2 \\
\beta &= \dfrac{3}{Jz} 
\end{align*}

Obs: El t\'ermino de orden 6 no es positivo en todo el espacio de par\'ametros, pero s\'i lo es en el entorno del punto tricr\'itico, por lo que el desarrollo anterior es v\'alido.

\pagebreak

\textbf{Problema 2: Modelo de Blume-Capel - Versi\'on Curie-Weiss}
\\


\begin{equation}
H = -\dfrac{J}{2N} \sum_{i,j} s_i s_j + D \sum_i s_i^2 - B\sum_i s_i
\end{equation}

Reescribimos el hamiltoniano como

\begin{equation}
H = -\dfrac{J}{2N} \left(\sum_i s_i \right)^2 + \dfrac{J}{2} + D \sum_i s_i^2 - B\sum_i s_i.
\end{equation}

La funci\'on de partici\'on es entonces 

\begin{equation}
\mathcal{Z} = e^{-K_1/2} \sum_{\lbrace s \rbrace} \exp \left[ \dfrac{K_1}{2N} \left(\sum_i s_i \right)^2 - K_2 \sum_i s_i^2 + h \sum_i s_i \right],
\end{equation}

donde $K_1 = \beta J$, $K_2 = \beta D$ y $h = \beta B$. Definimos 

\begin{equation}
a = \sqrt{\dfrac{K_1}{N}} \sum_i s_i
\end{equation}

y utilizamos la identidad gaussiana

\begin{equation}
e^{a^2/2} = \dfrac{1}{\sqrt{2\pi}}\int_{-\infty}^{\infty} e^{-x^2/2+ax} dx.
\end{equation}

Obtenemos

\begin{align}
\mathcal{Z} &= e^{-K_1/2} \sum_{\lbrace s \rbrace} \dfrac{1}{\sqrt{2\pi}}  \int_{-\infty}^{\infty} e^{-x^2/2} \exp \left[ - K_2 \sum_i s_i^2 + \left(x\sqrt{\dfrac{K_1}{N}} + h\right) \sum_i s_i \right] dx \nonumber \\
&= e^{-K_1/2} \dfrac{1}{\sqrt{2\pi}} \int_{-\infty}^{\infty} e^{-x^2/2} \left[1+  2 e^{-K_2}\cosh\left( x\sqrt{\dfrac{K_1}{N}} + h \right)\right]^N dx \nonumber \\
&= e^{-K_1/2} \dfrac{1}{\sqrt{2\pi}} \int_{-\infty}^{\infty} \exp\left\lbrace -\dfrac{x^2}{2} + N \ln\left[1+  2 e^{-K_2}\cosh\left( x\sqrt{\dfrac{K_1}{N}} + h \right)\right] \right\rbrace dx.
\end{align}

Realizamos el cambio de variables $\eta = x(K_1 N)^{-1/2}$ y obtenemos

\begin{equation}
\mathcal{Z} = e^{-K_1/2} \sqrt{\dfrac{KN}{2\pi}} \int_{-\infty}^{\infty} e^{Ng(\eta)} d\eta,
\end{equation}

donde

\begin{equation}
g(\eta) = \dfrac{K_1}{2}\eta^2 + \ln \left[1+ 2e^{-K_2}\cosh\left( K_1\eta + h \right) \right].
\end{equation}

Usando el m\'etodo de Laplace en el l\'imite $N\rightarrow\infty$ resulta

\begin{equation}
-\beta f = \max_{\eta} g(\eta).
\end{equation}

Derivando e igualando a cero obtenemos la ecuaci\'on

\begin{equation}
 \eta_0 = \dfrac{2e^{-K_2}\sinh\left( K_1\eta_0 + h \right)}{1+2e^{-K_2}\cosh\left( K_1\eta_0 + h \right)}.
\end{equation}

La magnetizaci\'on del sistema resulta

\begin{equation}
m = -\dfrac{\partial f}{\partial B} = \dfrac{\partial (-\beta f)}{\partial h} = \dfrac{2e^{-K_2}\sinh\left( K_1\eta_0 + h \right)}{1+2e^{-K_2}\cosh\left( K_1\eta_0 + h \right)} = \eta_0,
\end{equation}

de donde se sigue que la magnetizaci\'on satisface la relaci\'on

\begin{equation}
m = \dfrac{2e^{-K_2}\sinh\left( K_1 m + h \right)}{1+2e^{-K_2}\cosh\left( K_1 m + h \right)}.
\end{equation}

\pagebreak

\textbf{Problema 3: Teor\'ia de Landau para puntos tricr\'iticos}
\\

Consideremos un potencial de Landau gen\'erico para un sistema con simetr\'ia de paridad

\begin{equation}
\mathcal{L} = \dfrac{1}{2} a_2(T,x) \eta^2 + \dfrac{1}{2} a_4(T,x) \eta^4 + \dfrac{1}{6} a_6(T,x) \eta^6 - B \eta,
\end{equation}

donde $T$ es la temperatura, $B$ el campo magn\'etico, y $x$ el resto de los par\'ametros del sistema. Por simplicidad, consideramos $x$ unidimensional. Suponemos que $a_6(T,x) >0$, $\forall T,x$, lo cual justifica haber truncado el potencial a ese orden.

El punto tricr\'itico est\'a dado por la condici\'on $a_2 = a_4 = 0$. En un entorno de este punto, podemos escribir

\begin{align}
a_2(T,x) &= at + b\omega \label{eq:a2}\\
a_4(T,x) &= ct + d\omega, \label{eq:a4}
\end{align}

donde los coeficientes $a$, $b$, $c$ y $d$ son positivos y donde $t = (T - T_t)/T_t$ y $\omega = (x-x_t)/x_t$. Para obtener los exponentes para el punto tricr\'itico, nos aproximamos por la l\'inea $a_4 = 0$. En ese caso,

\begin{equation}
\omega = -\dfrac{c}{d}t,
\end{equation}

por lo que

\begin{equation}
a_2(t) = \left( a - \dfrac{c}{d}\right) t = A t,\quad A =  \left( a - \dfrac{c}{d}\right).
\end{equation}

Luego, el potencial de Landau adquiere la forma

\begin{equation}
\mathcal{L} = \dfrac{1}{2} A t \eta^2 + \dfrac{1}{6} a_6 \eta^6 - B \eta,
\end{equation}

donde suponemos $a_6$ constante en el entorno del PTC. La ecuaci\'on de estado correspondiente es 

\begin{equation}\label{eq:Landau_state}
At\eta + a_6\eta^5 = B.
\end{equation}

Para $B=0$, tenemos

\[ \eta = \begin{cases} 
      0 & \text{si } t>	 0 \\
      \left(\dfrac{A}{a_6}\right)^{1/4} (-t)^{1/4} &\text{si } t< 0,
   \end{cases}
\]

de donde $\beta_t = 1/4$.

Derivando \ref{eq:Landau_state} respecto a $B$,

\begin{equation}
A t \chi + 5 a_6 \eta^4 \chi = 1.
\end{equation}

Para $t>0$, tenemos $\eta = 0$, por lo que $\chi = A^{-1} t^{-1}$. Por otro lado, para $t<0$,

\begin{equation}
\chi = \dfrac{1}{At + 5a_6\eta^4} = \dfrac{1}{4A} (-t)^{-1}.
\end{equation}

Se deduce de aqu\'i que el exponente de la susceptibilidad es el mismo independientemente de la fase de la cual nos acercamos, con valor $\gamma_t = 1$.

Por otro lado, evaluando \ref{eq:Landau_state} en $t = 0$, tenemos

\begin{equation}
\eta = \dfrac{1}{a_6} B^{1/5},
\end{equation}

por lo que $\delta_t = 5$.

Por \'ultimo, recordemos que la energ\'ia libre $f$ es igual al m\'inimo del potencial $\mathcal{L}$. En la fase desordenada, $f = 0$, mientras que en la fase ordenada, tenemos, a campo nulo,

\begin{align}
f(t) &= -\dfrac{1}{2} A \left( \dfrac{A}{a_6}\right)^{1/2} t^{3/2} - \dfrac{1}{6} a_6 t^{3/2}\\
&= -\dfrac{1}{6}a_6\left[3 \left( \dfrac{A}{a_6}\right)^{3/2} + 1\right] t^{3/2}.
\end{align}

El calor espec\'ifico est\'a dado por

\begin{equation}
C = -T \dfrac{\partial^2 f}{\partial t^2} = \dfrac{1}{8} a_6\left[3 \left( \dfrac{A}{a_6}\right)^{3/2} + 1\right] (-t)^{-1/2},
\end{equation}

por lo que $\alpha_t = 1/2$.\\

\textbf{(b)} Aproximandonos al punto cr\'itico por la l\'inea de segundo orden tenemos que $a_4(T, x) = 0$. Luego, de \ref{eq:a4}, tenemos 

\begin{equation}
t = -\dfrac{b}{a} \omega \quad \Rightarrow \quad \dfrac{dt}{d\omega} = -\dfrac{b}{a}.
\end{equation}

Aproximandonos por la l\'inea de primer orden, tenemos que $(-a_4)^2 = a_2 a_6$. Derivando respecto a $\omega$ (considerando $a_6 = \text{cte} \neq 0$), tenemos

\begin{equation}
-6 \left( ct + d\omega \right) \left( c \dfrac{dt}{d\omega} + d \right) = \left(a\dfrac{dt}{d\omega} + b \right) a_6.
\end{equation}

Evaluando en $t = \omega = 0$, la ecuaci\'on anterior se cumple ssi $dt/d\omega = -b/a$. Luego, las pendientes de las curvas de primer y segundo orden en el plano $t\text{-}\omega$ y, por lo tanto en el plano $T\text{-}x$, son iguales.


\pagebreak

\textbf{Problema 4: Derivada funcional}\\

Sea $f(\mathbf{r})$ una funci\'on. Podemos interpretar a $f(\mathbf{r_1})$ como una funcional $F[f]$ de la forma

\begin{equation}
F[f] := \int f(\mathbf{r}) \delta(\mathbf{r} - \mathbf{r_1}) d \mathbf{r} = f(\mathbf{r_1}).
\end{equation}

Usando la definici\'on de la derivada funcional,

\begin{align*}
\dfrac{\delta f(\mathbf{r_1})}{\delta f(\mathbf{r_0})} &= \dfrac{\delta F}{\delta f(\mathbf{r_0})} \\
&= \lim_{\epsilon\rightarrow 0} \dfrac{\int \left[ f(\mathbf{r}) + \epsilon \delta(\mathbf{r} - \mathbf{r_0})\right] \delta(\mathbf{r} - \mathbf{r_1}) d \mathbf{r} - \int f(\mathbf{r}) \delta(\mathbf{r} - \mathbf{r_1}) d \mathbf{r}}{\epsilon} \\
&= \int \delta(\mathbf{r} - \mathbf{r_0}) \delta(\mathbf{r} - \mathbf{r_1}) d \mathbf{r} \\
&=  \delta(\mathbf{r_1} - \mathbf{r_0}).
\end{align*}

\begin{align*}
\dfrac{\delta}{\delta \eta(\mathbf{r})}\int \frac{1}{2} \left(\nabla \eta(\mathbf{r'})\right)^2 d\mathbf{r'} &= \lim_{\epsilon \rightarrow 0} \dfrac{ \int \frac{1}{2} \left[\nabla \left(\eta(\mathbf{r'}) + \epsilon \delta(\mathbf{r}-\mathbf{r'})\right)\right]^2 d\mathbf{r'} - \int \dfrac{1}{2} \left(\nabla \eta(\mathbf{r'})\right)^2 d\mathbf{r'} }{\epsilon} \\
&=   \int \nabla \eta(\mathbf{r'})\nabla  \delta(\mathbf{r}-\mathbf{r'}) d\mathbf{r'} \\
&= - \int \nabla^2 \eta(\mathbf{r'}) \delta(\mathbf{r}-\mathbf{r'}) d\mathbf{r'} \\
&= -\nabla^2 \eta(\mathbf{r})
\end{align*}

Sea $F[h] = \exp\left[\int h(\mathbf{r'})\eta(\mathbf{r'}) d\mathbf{r'}\right]$, entonces

\begin{align*}
\dfrac{\delta}{\delta h(\mathbf{r})} F[h] =& \lim_{\epsilon\rightarrow 0} \dfrac{1}{\epsilon}\bigg\lbrace 
\exp\left[\int (h(\mathbf{r'}) + \epsilon \delta(\mathbf{r}-\mathbf{r'}) \eta(\mathbf{r'}) d\mathbf{r'}\right] - \\
&-\exp\left[\int h(\mathbf{r'})\eta(\mathbf{r'}) d\mathbf{r'}\right] \bigg\rbrace \\
=& F[h] \lim_{\epsilon\rightarrow 0} \dfrac{1}{\epsilon}  \bigg\lbrace \exp\bigg[\epsilon \int \delta(\mathbf{r} - \mathbf{r'})\eta(\mathbf{r'}) d\mathbf{r'}\bigg] -1 \bigg\rbrace.
\end{align*}

Desarrollando en potencias la exponencial dentro del l\'imite,

\begin{align*}
\dfrac{\delta}{\delta h(\mathbf{r})} F[h] 
&= F[h] \lim_{\epsilon\rightarrow 0} \dfrac{1}{\epsilon}  \bigg\lbrace 1+ \epsilon \int \delta(\mathbf{r} - \mathbf{r'})\eta(\mathbf{r'}) d\mathbf{r'} - 1\bigg\rbrace \\
&= F[h] \int \delta(\mathbf{r} - \mathbf{r'})\eta(\mathbf{r'}) d\mathbf{r'} \\
&= \eta(\mathbf{r}) F[h] 
\end{align*}

\pagebreak

\textbf{Problema 5: Funcional de Landau-Ginzburg}\\

Usando la propiedad del problema anterior, tenemos que 

\begin{equation}
\langle \eta(\mathbf{r})\rangle \equiv \dfrac{1}{\mathcal{Z}} \int \mathcal{D}\eta\; \eta(\mathbf{r'})  e^{-\beta L[\eta]} = \dfrac{1}{\beta} \dfrac{\delta \ln \mathcal{Z}}{\delta B(\mathbf{r})}.
\end{equation}

Por otro lado

\begin{align*}
\chi(\mathrm{r}, \mathrm{r'}) &\equiv \dfrac{\delta \langle \eta(\mathbf{r}) \rangle}{\delta B(\mathbf{r'})} \\
&= \dfrac{1}{\beta} \dfrac{\delta}{\delta B(\mathbf{r'})}\left[ \dfrac{1}{\mathcal{Z}} \dfrac{\delta \mathcal{Z}}{\delta B(\mathbf{r})} \right] \\
&= \dfrac{1}{\beta} \left[ \dfrac{1}{\mathcal{Z}} \dfrac{\delta \delta \mathcal{Z}}{\delta B(\mathbf{r}) \delta B(\mathbf{r'})} - \dfrac{1}{\mathcal{Z}^2} \dfrac{\delta \mathcal{Z}}{\delta B(\mathbf{r})} \dfrac{\delta \mathcal{Z}}{\delta B(\mathbf{r'})}  \right] \\
&= \beta \bigg[ \langle \eta(\mathbf{r})  \eta(\mathbf{r'}) \rangle - \langle  \eta(\mathbf{r}) \rangle \langle  \eta(\mathbf{r'}) \rangle \bigg] 
\end{align*}

\end{document}
