\documentclass[10pt]{article}
\usepackage{amsmath} 

\begin{document}

\textbf{Problema 1: Modelo de Blume-Capel - Aproximaci\'on variacional}
\\

\begin{equation}
H = -J \sum_{\langle i,j\rangle} s_i s_j + D \sum_i s_i^2
\end{equation}

\begin{equation}
H_0 = -\eta \sum_i s_i + D \sum_i s_i^2
\end{equation}

Desigualdad de Bogoliuvob-Peierls:

\begin{equation}
f \leq f_{\rho} = f_0 + \dfrac{1}{N} \langle H - H_0 \rangle_0.
\end{equation}

Funci\'on de partici\'on para el hamiltoniano de prueba:

\begin{align}
\mathcal{Z}_0 &= \sum_{\lbrace s_i\rbrace} e^{-\beta H_0} \nonumber \\
&= \sum_{\lbrace s_i\rbrace} \exp \left[ \beta \eta \sum_i s_i - \beta D \sum_i s_i^2 \right] \nonumber \\
&= \sum_{\lbrace s_i\rbrace} \prod_i  \exp \left[ \beta \eta s_i - \beta D  s_i^2 \right] \nonumber \\
&= \prod_i \sum_{s_i=0,\pm1} \exp \left[ \beta \eta s_i - \beta D  s_i^2 \right] \nonumber \\
&= \left[1 + e^{-\beta \eta -\beta D} + e^{\beta \eta -\beta D} \right]^N \nonumber \\
&= \left[1 + 2e^{-\beta D} \cosh\left(\beta \eta\right) \right]^N, \nonumber \\
&= \mathcal{Z}_{01}^N,
\end{align}

donde 

\begin{equation}
\mathcal{Z}_{01} = 1 + 2e^{-\beta D} \cosh\left(\beta \eta\right).
\end{equation}

Energ\'ia libre:

\begin{align}
f_0 &= -\dfrac{1}{\beta N} \ln \mathcal{Z}_0 \nonumber \\
&= -\dfrac{1}{\beta} \ln \left[ 1 + 2e^{-\beta D} \cosh\left(\beta \eta\right) \right]
\end{align}

Magnetizaci\'on:

\begin{align} \label{eq:BC_m0}
m_0 &= \langle s_i \rangle_0 \nonumber \\
&= \dfrac{1}{\mathcal{Z}_{01}} \sum_{s_i=0,\pm1} s_i e^{\beta \eta s_i - \beta D  s_i^2} \nonumber \\
&= \dfrac{1}{\mathcal{Z}_{01}} \left[-e^{-\beta \eta -\beta D} + e^{\beta \eta -\beta D} \right] \nonumber \\
&= \dfrac{2e^{-\beta \eta}\sinh\left( \beta \eta \right)}{1+2e^{-\beta \eta}\cosh\left( \beta \eta \right)}
\end{align}

Segundo momento:

\begin{align}
\langle s_i^2 \rangle_0 &= \dfrac{1}{\mathcal{Z}_{01}} \sum_{s_i=0,\pm1} s_i^2 e^{\beta \eta s_i - \beta D  s_i^2} \nonumber \\
&= \dfrac{1}{\mathcal{Z}_{01}} \left[e^{-\beta \eta -\beta D} + e^{\beta \eta -\beta D} \right] \nonumber \\
&= \dfrac{2e^{-\beta \eta}\cosh\left( \beta \eta \right)}{1+2e^{-\beta \eta}\cosh\left( \beta \eta \right)}
\end{align}

Valores medios de los hamiltonianos respecto del hamiltoniano de prueba:

\begin{align}
\langle H_0 \rangle_0 &= -\eta \sum_i \langle_0 s_i \rangle + D \sum_i \langle s_i^2 \rangle_0 \nonumber \\
&= -\eta N m_0 + D N \langle s_i^2 \rangle_0 \nonumber
\end{align}

\begin{align}
\langle H \rangle_0 &= -J \sum_{\langle i,j\rangle} \langle s_i s_j \rangle_0 + D \sum_i \langle s_i^2 \rangle_0 \nonumber \\
&= -J \sum_{\langle i,j\rangle} \langle s_i \rangle_0 \langle s_j \rangle_0 + D \sum_i \langle s_i^2 \rangle_0 \nonumber \\
&= -J \dfrac{Nz}{2} m_0^2 + D N \langle s_i^2 \rangle_0 \nonumber
\end{align}

Proponemos la funci\'on variacional 

\begin{align} \label{eq:BC_Phi}
\Phi(\eta) &= f_0 + \dfrac{1}{N} \langle H - H_0 \rangle_0 \nonumber \\
&= -\dfrac{1}{\beta} \ln \left[ 1 + 2e^{-\beta D} \cosh\left(\beta \eta\right) \right] - J \dfrac{z}{2} m_0^2 + \eta m_0
\end{align}

Derivamos con respecto a $\eta$ e igualamos a cero para hallar el m\'inimo

\begin{align}
\dfrac{\partial \Phi}{\partial \eta} &= -  \dfrac{2e^{-\beta \eta}\sinh\left( \beta \eta \right)}{1+2e^{-\beta \eta}\cosh\left( \beta \eta \right)} - Jzm_0 \dfrac{\partial m_0}{\partial \eta} + m_0 + \eta \dfrac{\partial m_0}{\partial \eta} \nonumber \\
&= \left(\eta - Jzm_0\right) \dfrac{\partial m_0}{\partial \eta},
\end{align}

donde usamos la igualdad \ref{eq:BC_m0}.

La expresi\'on anterior implica que la soluci\'on al problema variacional es 

\begin{equation}
\eta = Jzm_0.
\end{equation}

Reemplazando en \ref{eq:BC_Phi}, tenemos

\begin{equation}
\Phi(\eta) = -\dfrac{1}{\beta} \ln \left[ 1 + 2e^{-\beta D} \cosh\left(\beta \eta\right) \right]  + \dfrac{1}{2Jz} \eta^2
\end{equation}

Realizamos un desarrollo de Taylor de orden 4 de la expresi\'on anterior, teniendo en cuenta la siguiente relaci\'on:

\begin{equation}
\ln \left[ 1+ c \cosh(x) \right] = \ln(c+1) + \dfrac{c}{2c+1} x^2 + \dfrac{c(1-2c)}{24(c+1)^2} x^4
\end{equation}

\end{document}
