\documentclass[10pt]{article}
\usepackage{amsmath} 

\begin{document}

\textbf{Problema 1: Gas de Van der Waals}
\\

Ecuaci\'on de estado:

\begin{equation}
\left(p+\dfrac{a}{v^2}\right)\left( v-b\right)= RT.
\end{equation}

(a) La transici\'on de fase ocurre cuando 

\begin{align}
\dfrac{\partial p}{\partial v} &= 0\\
\dfrac{\partial^2 p}{\partial v^2} &= 0
\end{align}

Despejando la presi\'on,

\begin{equation} \label{eq:VdW_pVsRest}
p = \dfrac{RT}{v-b} - \dfrac{a}{v^2}.
\end{equation}

Derivando una vez e igualando a cero

\begin{equation} \label{eq:VdW_deriv}
\dfrac{\partial p}{\partial v} = -\dfrac{RT}{(v-b)^2}+ \dfrac{2a}{v^3} = 0\quad \Leftrightarrow \quad RTv^3 = 2a(v-b)^2.
\end{equation}

Derivando nuevamente

\begin{equation} \label{eq:VdW_deriv2}
\dfrac{\partial^2 p}{\partial v^2} = \dfrac{2RT}{(v-b)^3} - \dfrac{6a}{v^4} = 0\quad \Leftrightarrow \quad 2RTv^4 = 6a(v-b)^3.
\end{equation}

Dividiendo \ref{eq:VdW_deriv2} por \ref{eq:VdW_deriv},

\begin{equation}
2v = 3(v-b)\quad \Leftrightarrow \quad v_c = 3b.
\end{equation}

Reemplazando $v_c$ en \ref{eq:VdW_deriv2},

\begin{equation}
T_c = \dfrac{2a (2b)^2}{R(3b)^3} = \dfrac{8}{27}\dfrac{a}{Rb}
\end{equation}

y reemplazando $T_c$ y $v_c$ en \ref{eq:VdW_pVsRest},

\begin{equation}
p_c = \dfrac{8}{27} \dfrac{a}{b} \dfrac{1}{2b} - \dfrac{a}{9b^2} = \dfrac{a}{27 b^2}.
\end{equation}

Una relaci\'on que resulta \'util es 

\begin{equation}
\dfrac{p_c v_c}{T_c} = \dfrac{3}{8}R.
\end{equation}

(b) Partiendo de \ref{eq:VdW_pVsRest},

\begin{align}
(\pi + 1)p_c &= \dfrac{R(t+1)T_c}{(\omega + 1)v_c-b} - \dfrac{a}{(\omega+1)^2v_c^2} \\
\pi + 1 &= \dfrac{(t+1)(RT_c/p_cv_c)}{(\omega + 1)-b/v_c} - \dfrac{a}{(\omega+1)^2p_cv_c^2} \\
\pi + 1 &= \dfrac{8(t+1)}{3(\omega + 1)-1} - \dfrac{3}{(\omega+1)^2}\\
\pi &= \dfrac{4(t+1)}{\frac{3}{2}\omega + 1} - \dfrac{3}{(\omega+1)^2} - 1
\end{align}

\end{document}