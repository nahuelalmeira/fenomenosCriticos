\documentclass[10pt]{article}
\usepackage{amsmath} 

\begin{document}

\textbf{Problema 1: Gas de Van der Waals}
\\

Ecuaci\'on de estado:

\begin{equation}
\left(p+\dfrac{a}{v^2}\right)\left( v-b\right)= RT.
\end{equation}

(a) La transici\'on de fase ocurre cuando 

\begin{align}
\dfrac{\partial p}{\partial v} &= 0\\
\dfrac{\partial^2 p}{\partial v^2} &= 0
\end{align}

Despejando la presi\'on,

\begin{equation} \label{eq:VdW_pVsRest}
p = \dfrac{RT}{v-b} - \dfrac{a}{v^2}.
\end{equation}

Derivando una vez e igualando a cero

\begin{equation} \label{eq:VdW_deriv}
\dfrac{\partial p}{\partial v} = -\dfrac{RT}{(v-b)^2}+ \dfrac{2a}{v^3} = 0\quad \Leftrightarrow \quad RTv^3 = 2a(v-b)^2.
\end{equation}

Derivando nuevamente

\begin{equation} \label{eq:VdW_deriv2}
\dfrac{\partial^2 p}{\partial v^2} = \dfrac{2RT}{(v-b)^3} - \dfrac{6a}{v^4} = 0\quad \Leftrightarrow \quad 2RTv^4 = 6a(v-b)^3.
\end{equation}

Dividiendo \ref{eq:VdW_deriv2} por \ref{eq:VdW_deriv},

\begin{equation}
2v = 3(v-b)\quad \Leftrightarrow \quad v_c = 3b.
\end{equation}

Reemplazando $v_c$ en \ref{eq:VdW_deriv2},

\begin{equation}
T_c = \dfrac{2a (2b)^2}{R(3b)^3} = \dfrac{8}{27}\dfrac{a}{Rb}
\end{equation}

y reemplazando $T_c$ y $v_c$ en \ref{eq:VdW_pVsRest},

\begin{equation}
p_c = \dfrac{8}{27} \dfrac{a}{b} \dfrac{1}{2b} - \dfrac{a}{9b^2} = \dfrac{a}{27 b^2}.
\end{equation}

Una relaci\'on que resulta \'util es 

\begin{equation}
\dfrac{p_c v_c}{T_c} = \dfrac{3}{8}R.
\end{equation}

(b) Partiendo de \ref{eq:VdW_pVsRest},

\begin{align}
(\pi + 1)p_c &= \dfrac{R(t+1)T_c}{(\omega + 1)v_c-b} - \dfrac{a}{(\omega+1)^2v_c^2} \\
\pi + 1 &= \dfrac{(t+1)(RT_c/p_cv_c)}{(\omega + 1)-b/v_c} - \dfrac{a}{(\omega+1)^2p_cv_c^2} \\
\pi + 1 &= \dfrac{8(t+1)}{3(\omega + 1)-1} - \dfrac{3}{(\omega+1)^2}\\
\pi &= \dfrac{4(t+1)}{\frac{3}{2}\omega + 1} - \dfrac{3}{(\omega+1)^2} - 1
\end{align}

(c) Para hacer la aproximaci\'on, defino la funci\'on

\begin{equation}
f(\pi, \omega, t) = \dfrac{4(1+t)}{1+\frac{3}{2}\omega} - \dfrac{3}{(1+\omega)^2} - \pi - 1
\end{equation}

y realizo una expansi\'on de Taylor en torno a $(\pi, \omega, t) = (0, 0, 0)$.

\begin{align}
\dfrac{\partial f}{\partial \pi} &= -1 \\
\dfrac{\partial f}{\partial \omega} &= -\dfrac{6(1+t)}{1+\frac{3}{2}\omega}+ \dfrac{6}{(1+\omega)^3} \\
\dfrac{\partial f}{\partial t} &= \dfrac{4}{1+\frac{3}{2}\omega} \\
\dfrac{\partial^2 f}{\partial t \partial\omega} &= -\dfrac{6}{(1+\frac{3}{2}\omega)^2} \\
\dfrac{\partial^2 f}{\partial \omega^2} &= 18\left[\dfrac{(1+t)}{(1+\frac{3}{2}\omega)^3} - \dfrac{1}{(1+\omega)^4}\right]\\
\dfrac{\partial^3 f}{\partial \omega^3} &= 18\left[-\dfrac{9}{2}\dfrac{(1+t)}{(1+\frac{3}{2}\omega)^4} + \dfrac{4}{(1+\omega)^5}\right]
\end{align}

Teniendo en cuenta que las derivadas de orden superior con respecto a $\pi$ se anulan, la expresi\'on queda

\begin{align}
f(\pi, \omega, t) &= f(0, 0, 0) \nonumber\\ 
&+ \dfrac{\partial f}{\partial \pi} \Bigr|_{0} \pi + \dfrac{\partial f}{\partial \omega}\Bigr|_{0} \omega + \dfrac{\partial f}{\partial t}\Bigr|_{0} t \nonumber\\
&+ \dfrac{1}{2} \dfrac{\partial^2 f}{\partial \omega^2}\Bigr|_{0} \omega^2 + \dfrac{\partial^2 f}{\partial t\omega}\Bigr|_{0} t\omega + \dfrac{1}{6} \dfrac{\partial^3 f}{\partial \omega^3}\Bigr|_{0} \omega^3 \nonumber \\
&+\mathcal{O}(t\omega^2, \omega^4).
\end{align}

Evaluando las derivadas, se obtiene

\begin{equation}
f(\pi, \omega, t) = -\pi + 4t -6t\omega -\dfrac{3}{2}\omega^3 + \mathcal{O}(t\omega^2, \omega^4)
\end{equation}

Luego,

\begin{equation}\label{eq:pi_approx}
\pi = 4t -6t\omega -\dfrac{3}{2}\omega^3  + \mathcal{O}(t\omega^2, \omega^4)
\end{equation}

d) La construcci\'on de Maxwell implica la igualdad

\begin{equation}
\int_{v_l}^{v_g} p dv = p_0 (v_g - v_l),
\end{equation}

donde $p_0 = p(v_g) = p(v_l)$.

Utilizando las variables reducidas, se tiene que 

\begin{align}
\int_{v_l}^{v_g} p dv &= p_0 (v_g - v_l) \\
\int_{\omega_1}^{\omega_2} (1+\pi) p_c v_c d\omega &= (1+\pi_0) p_c (\omega_2 - \omega_1) v_c \\
\int_{\omega_1}^{\omega_2} \pi d\omega &= \pi_0 (\omega_2 - \omega_1),
\end{align}

donde $\pi_0 = \pi_0(\omega_2) = \pi_0(\omega_1)$

Utilizando la expresi\'on aproximada dada por la ecuaci\'on \ref{eq:pi_approx} e integrando,

\begin{align}
4t(\omega_2 - \omega_1) - 3t(\omega_2^2 - \omega_1^2) - \dfrac{3}{8} (\omega_2^4- \omega_1^4) &= \pi_0 (\omega_2 - \omega_1) \nonumber \\
4t - 3t(\omega_2 + \omega_1) - \dfrac{3}{8} (\omega_2 + \omega_1) (\omega_2^2 + \omega_2 \omega_1 + \omega_1^2) &= \pi_0. \label{eq:maxwell_integrated}
\end{align}

Por otro lado,

\begin{align}
\pi_0(\omega_1) &= \pi_0(\omega_2) \nonumber \\
 -6t\omega_1 - \dfrac{3}{2}\omega_1^3 &= -6t\omega_2 - \dfrac{3}{2}\omega_2^3 \nonumber \\
 2t(\omega_2 - \omega_1) + \dfrac{1}{2} (\omega_2^3-\omega_1^3 ) &= 0\nonumber \\
  4t + \omega_2^2+\omega_2  \omega_1 + \omega_1^2 &= 0. \label{eq:4t_equal_omegas}
\end{align}

Reemplazando \ref{eq:4t_equal_omegas} en \ref{eq:maxwell_integrated},

\begin{align}
4t - 3t(\omega_2 + \omega_1) + \dfrac{3}{2} t (\omega_2 + \omega_1) &= \pi_0 \nonumber \\
4t - \dfrac{3}{2} t (\omega_2 + \omega_1) &= \pi_0
\end{align}

Evaluando $\pi_0$ en $\omega_2$, se tiene

\begin{align}
 -\dfrac{3}{2} t (\omega_2 + \omega_1) &= -6t\omega_2 -\dfrac{3}{2}\omega_2^3 \nonumber \\
t (\omega_2 + \omega_1) &= 4t\omega_2 + \omega_2^3 \nonumber \\
t \omega_1  &= 3 t \omega_2 + \omega_2^3 \nonumber \label{eq:t_omega1}
\end{align}

Reemplazando $\omega_1$ en  \ref{eq:4t_equal_omegas} y conservando s\'olo los \'ordenes relevantes,

\begin{align}
  \omega_2^2+\omega_2 \left( 3 \omega_2 - \dfrac{\omega_2^3}{t} \right) + \left(3 \omega_2 - \dfrac{\omega_2^3}{t} \right)^2 &= 4t \nonumber \\
  4 \omega_2^2 - \dfrac{\omega_2^4}{t} + 9 \omega_2^2 - 6 \dfrac{\omega_2^4}{t} + \left( \dfrac{\omega_2^3}{t} \right)^2 &= 4t 
\end{align}

TODO: Continuar (y verificar)
\\

\textbf{Problema 3: Curie-Weiss}
\\

\begin{equation} \label{eq:Curie_Weiss}
m = \tanh (\beta B + \beta \lambda m)
\end{equation}

Consideremos primero el caso de campo nulo. En ese caso, tenemos que 

\begin{equation} \label{eq:Curie_Weiss_B0}
m = \tanh (\beta \lambda m)
\end{equation}

El punto cr\'itico est\'a dado por la condici\'on de que el argumento de la tangente hiperb\'olica sea igual a $m$, es decir, que

\begin{align}
\beta_c &= \dfrac{1}{\lambda} \nonumber \\
k_B T_c &= \lambda \nonumber
\end{align}

Para ver c\'omo escala la magnetizaci\'on con respecto a la temperatura, desarrollamos la tangente en torno al punto cr\'itico.

\begin{align}
m &= \beta \lambda m - \dfrac{(\beta \lambda m)^3}{3} \nonumber \\
1 &= \beta \lambda - \dfrac{\beta^3 \lambda^3 m^2}{3}\nonumber \\
\dfrac{\beta^3 \lambda^3 m^2}{3} &= \beta \lambda - 1\nonumber \\
\dfrac{\beta^2 \lambda^2 m^2}{3} &= 1- \dfrac{1}{\beta \lambda}\nonumber \\
\dfrac{\beta^2 \lambda^2 m^2}{3} &= 1- \dfrac{k_B T}{ \lambda} \nonumber\\
\dfrac{\beta^2 \lambda^2 m^2}{3} &= 1- \dfrac{T}{ T_c}\nonumber \\
m &= \left[\dfrac{3}{\beta^2 \lambda^2} (-t)\right]^{1/2} \label{eq:m_function_t}
\end{align}

La susceptibilidad magn\'etica se define como 

\begin{equation}
\chi(T, B) = \dfrac{\partial m}{\partial B}.
\end{equation}

Para obtener el escaleo con respecto a la temperatura de $\chi(T, B\rightarrow 0)$, aproximamos \ref{eq:Curie_Weiss} a orden c\'ubico y derivamos ambos miembros.

\begin{align}
m &= (\beta B + \beta \lambda m) - \dfrac{(\beta B + \beta \lambda m)^3}{3} \nonumber \\
\chi &= (\beta + \beta \lambda \chi) - (\beta B + \beta \lambda m)^2 (\beta + \beta \lambda \chi). 
\end{align}

A campo nulo, tenemos,

\begin{align}
\chi &= (\beta + \beta \lambda \chi) - (\beta \lambda m)^2 (\beta + \beta \lambda \chi). 
\end{align}

Para temperaturas superiores a la temperatura cr\'itica, no existe magnetizaci\'on espont\'anea, por lo que 

\begin{align}
\chi &= (\beta + \beta \lambda \chi) \nonumber \\
(1 - \beta \lambda )\chi &= \beta  \nonumber  \\
(k_B T - \lambda )\chi &= 1  \nonumber  \\
k_B (T -T_c )\chi &= 1 \nonumber   \\
\lambda \dfrac{(T -T_c )}{T_c}\chi &= 1  \nonumber  \\
\chi &= \dfrac{1}{\lambda} t^{-1}.
\end{align}

Por otro lado, si $T < T_c$, podemos utilizar la ecuaci\'on \ref{eq:m_function_t}

\begin{align}
\chi &= (\beta + \beta \lambda \chi) - 3 \left( 1 - \dfrac{1}{\beta \lambda} \right)(\beta + \beta \lambda \chi) \nonumber \\
\chi &= \beta - 3\beta  + \beta \lambda \chi - 3 \beta \lambda \chi + \dfrac{3}{\lambda} + 3\chi \nonumber \\
-2 \chi &= -2 \beta - 2\beta \lambda \chi + \dfrac{3}{\lambda} \nonumber \\
-2(1 - \beta \lambda ) \chi &= -2 \beta  + \dfrac{3}{\lambda} \nonumber \\
-2(k_B T -  \lambda ) \chi &= -2   + \dfrac{3k_B T}{\lambda} \nonumber \\
-2\lambda t \chi &= -2   + \dfrac{T}{T_c} \nonumber \\
\chi &= \dfrac{-2 + 3\tilde{T}}{2\lambda} (-t)^{-1},
\end{align}

donde definimos $\tilde{T} = T/T_c$. TODO: Los prefactores deber\'ian dar igules!

\end{document}