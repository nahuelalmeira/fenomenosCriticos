\documentclass[10pt]{article}
\usepackage{amsmath} 

\begin{document}

\textbf{Problema 1: Decimaci\'on}
\\

Modelo de Ising unidimencional:

\begin{equation}
\mathcal{H} = K \sum_i s_i s_{i+1} + h \sum_i s_i,
\end{equation}

donde $\mathcal{H} \equiv -\beta H$, $K\equiv \beta J$ y $h \equiv \beta B$.

Tomamos bloques de Kadanoff de dos sitios, de los cuales decimamos uno de los sitios. Llamamos $S_i$ a las variables de bloque, y $\sigma_i$ a las variables correspondientes a los sitios decimados. El hamiltoniano entonces puede ser escrito como

\begin{equation}
\mathcal{H} = K \sum_i \sigma_i (S_i + S_{i+1}) + \dfrac{h}{2} \sum_i (S_i + S_{i+1}) + h \sum_i \sigma_i.
\end{equation}

Suponemos que, al renormalizar, no hay proliferaci\'on de interacciones, por lo que podemos proponer un hamiltoniano renormalizado de la forma

\begin{equation}
\mathcal{H}' = K' \sum_i S_i S_{i+1} + \dfrac{h'}{2} \sum_i (S_i + S_{i+1}).
\end{equation}

Recordemos que, para el caso de decimaci\'on, la transformaci\'on de normalizaci\'on est\'a dada por 

\begin{equation}
e^{\mathcal{C} + \mathcal{H}'(S)} = \sum_{\lbrace \sigma\rbrace} e^{\mathcal{H}(S,\sigma)},
\end{equation}

donde $\mathcal{C} = Ng(K,h)$ es la constante de normalizaci\'on y $g(K,h)$ es una funci\'on intensiva de los par\'ametros.

En este caso,

\begin{equation}
e^{\mathcal{C} + \mathcal{H}'(S)} = \prod_i \exp\left[  K' S_i S_{i+1} + \dfrac{h'}{2} (S_i + S_{i+1}) + g \right]
\end{equation}

y

\begin{align}
\sum_{\lbrace \sigma \rbrace} e^{\mathcal{H}(S,\sigma)} &= \prod_i \sum_{\sigma=\pm 1} \exp\left[  K \sigma_i (S_i + S_{i+1}) + \dfrac{h}{2} (S_i + S_{i+1}) + h \sigma_i \right] \nonumber \\
&=  \prod_i 2 e^{\frac{h}{2} (S_i + S_{i+1}) }  \cosh\left[ K (S_i + S_{i+1}) + h \right] \nonumber
\end{align}

Igualando las dos expresiones anteriores, tenemos

\begin{equation}
 \exp\left[  K' S_i S_{i+1} + \dfrac{h'}{2} (S_i + S_{i+1}) + g \right] = e^{\frac{h}{2} (S_i + S_{i+1}) }  \cosh\left[ K (S_i + S_{i+1}) + h \right],
\end{equation}

para cada bloque $i$.

Evaluando para cada combinaci\'on de spines, tenemos las relaciones

\begin{align}
e^{g+K'+h'} &= 2e^{h}\cosh(2K+h) \label{eq:Ising1D_Rec1}\\
e^{g+K'-h'} &= 2e^{-h}\cosh(2K-h) \label{eq:Ising1D_Rec2} \\
e^{g-K'} &= 2\cosh(h)  \label{eq:Ising1D_Rec3}
\end{align}

En el caso $h=0$, tenemos que $h'=0$. Resolviendo para $K'$,

\begin{align}
K'(K) &= \dfrac{1}{2} \ln \left[ \cosh(2K) \right] \label{eq:Ising1D_Kprime}\\
g(K) &= \ln(2) + K'(K)
\end{align}

La expresi\'on \ref{eq:Ising1D_Kprime} se puede simplificar mediante el cambio de variables

\begin{align}
t  &= \tanh(K)\\
t' &= \tanh(K'),
\end{align}

y teniendo en cuenta la identidad 

\begin{equation}
z = \tanh\left[\dfrac{1}{2}\ln\left( \dfrac{z+1}{z-1} \right)\right],
\end{equation}

la cual implica que 

\begin{equation}
K = \dfrac{1}{2}\ln\left( \dfrac{t+1}{t-1} \right) \quad \Rightarrow \quad e^{2K} = \dfrac{t+1}{t-1} 
\end{equation}

De \ref{eq:Ising1D_Kprime}, tenemos que

\begin{align}
 t' = \tanh(K') = \tanh \left[ \dfrac{1}{2} \ln \left[ \cosh(2K) \right]  \right],
\end{align}

por lo que se deduce que 

\begin{align}
 \dfrac{t'+1}{t'-1}  &= \cosh(2K) \nonumber \\
 &= \dfrac{\dfrac{t+1}{t-1} + \dfrac{t-1}{t+1}}{2} \nonumber \\
 &= \dfrac{t^2+1}{(t^2-1)}.
\end{align}

Simplificando la expresi\'on anterior, obtenemos la relaci\'on de recurrencia

\begin{equation} \label{eq:Ising1D_tprime}
t' = t^2.
\end{equation}

Otra forma de llegar a la relaci\'on anterior es partiendo de 

\begin{equation}
e^{2K'} = \cosh(2K)
\end{equation}

y notando que 

\begin{equation}
\tanh^2(K) = \dfrac{e^{2K} + e^{-2K} + 2}{e^{2K} + e^{-2K} - 2} = \dfrac{\cosh(2K) + 1}{\cosh(2K) - 1}= \dfrac{e^{2K'} + 1}{e^{2K'} - 1} = \tanh(K') 
\end{equation}

La relaci\'on \ref{eq:Ising1D_tprime} indica que hay dos puntos fijos: $t^*=0$, que es estable y $t^*=1$, que es inestable. Estos corresponden a $K=0$ (temperatura infinita) y $K=\infty$ (temperatura cero). Es decir que el modelo precice una transici\'on de fase a temperatura id\'enticamente cero.

El exponente cr\'itico $\nu$, asociado a la divergencia de la longitud de correlaci\'on, puede calcularse mediante $\nu = y_T^{-1}$, donde 

\begin{equation}
y_T = \dfrac{\ln\left(\frac{\partial K'}{\partial K}\Bigr|_{K_c}\right)}{\ln (b)}.
\end{equation}

En este caso,

\begin{equation}
\frac{\partial K'}{\partial K}\Bigr|_{K_c} = \tanh(2K_c) = 1,
\end{equation}

por lo que $\nu = \infty$

Analizamos ahora el caso $h>0$. Definimos las variables $u=e^{-4K}$ y $v=e^{-2h}$. Dividiendo  \ref{eq:Ising1D_Rec2} por  \ref{eq:Ising1D_Rec1}, obtenemos la ecuaci\'on para $v'$

\begin{align}
e^{-2h'} &= e^{-2h} \dfrac{\cosh(2K-h)}{\cosh(2K+h)} \nonumber \\
v' &= v \dfrac{e^{2K-h} + e^{-2K+h}}{e^{2K+h} + e^{-2K-h}} \nonumber \\
v' &= v \dfrac{v+u}{1+uv}.
\end{align}

Elevando \ref{eq:Ising1D_Rec3} al cuadrado,

\begin{align}\label{eq:Ising1D_Rec4}
e^{2g-2K'} &= (e^h + e^{-h})^2 = \dfrac{(1+v)^2}{v} 
\end{align}

Multiplicando \ref{eq:Ising1D_Rec2} por  \ref{eq:Ising1D_Rec1},

\begin{align}
e^{2g+2K'} &= (e^{2K+h} + e^{-2K-h}) (e^{2K-h} + e^{-2K+h}) \nonumber \\
&= \dfrac{(1+uv)(v+u)}{uv}. \label{eq:Ising1D_Rec5}
\end{align}

Por \'ultimo, dividiendo \ref{eq:Ising1D_Rec4} por \ref{eq:Ising1D_Rec5}, obtenemos la ecuaci\'on para $u'$. Juntas, ambas son

\begin{align}
u' &= \dfrac{u (1+v)^2}{(1+uv)(v+u)} \\
v' &= v \dfrac{v+u}{1+uv}.
\end{align}

Por inspecci\'on, podemos ver que los puntos $(0,0)$ y $(1,1)$ son puntos fijos. Estos corresponden a las fases ordenada $(T = 0)$ y desordenada $(T = \infty)$, respectivamente. Adem\'as, si $v^*=1$, que corresponder\'ia a campo nulo, tenemos

\begin{align}
u^* &= \dfrac{4u^*}{(1+u^*)^2},
\end{align}

cuyas soluciones son $u^* = 0$ y $u^* = 1$. Por lo tanto $(0,1)$ es tambi\'en un punto fijo. Por otro lado, si $u^* = 1$, cualquier valor de $v$ entre 0 y 1 es soluci\'on, por lo que $(1,v)$ es una l\'inea c\'itica.

\pagebreak

\textbf{Problema 2: Aproximaci\'on de Migdal-Kadanoff}\\

Antes de utilizar la aproximaci\'on de M-K, necesitamos la ecuaci\'on de transformaci\'on para decimaci\'on de una cadena lineal con $b = 3$.

\begin{align*}
\exp \left[ K' S_1 S_2 + \dfrac{h}{2}\left( S_1 + S_2\right) +g \right] = \sum_{\lbrace \sigma \rbrace} \exp &\bigg[ K(S_1 \sigma_1 + \sigma_1 \sigma_2 + \sigma_2 S_2) \\
&+ h (\sigma_1 + \sigma_2) + \dfrac{h}{2} (S_1 + S_2) \bigg]
\end{align*}

\begin{align*}
&\sum_{\lbrace \sigma \rbrace} \exp \bigg[ K(S_1 \sigma_1 + \sigma_1 \sigma_2 + \sigma_2 S_2) + h (\sigma_1 + \sigma_2) + \dfrac{h}{2} (S_1 + S_2) \bigg] = \\
&e^{\frac{h}{2} (S_1 + S_2)} \sum_{\sigma_1} \sum_{\sigma_2} \exp \bigg[ K(S_1 \sigma_1 + \sigma_1 \sigma_2 + \sigma_2 S_2) + h (\sigma_1 + \sigma_2) \bigg] = \\
&e^{\frac{h}{2} (S_1 + S_2)} \bigg[ e^{K(-S_1 + 1 - S_2) - 2h} + e^{K(-S_1 - 1 +S_2)} + e^{K(S_1 - 1 - S_2)} + e^{K(S_1 + 1 + S_2) + 2h} \bigg] = \\
&2 e^{\frac{h}{2} (S_1 + S_2)}\bigg\lbrace e^{K} \cosh\left[ K(S_1+S_2)+2h \right] + e^{-K} \cosh\left[ K(S_1-S_2) \right] \bigg\rbrace 
\end{align*}

Igualando para combinaci\'on de $S_1$ y $S_2$,

\begin{align}
e^{g+K'+h'} &= 2e^{h}\left[e^{-K} + e^K\cosh(2K+2h) \right] \label{eq:b3_Rec1}\\
e^{g+K'-h'} &= 2e^{-h}\left[e^{-K} + e^K\cosh(2K-2h) \right]\label{eq:b3_Rec2} \\
e^{g-K'} &= 2\left[ e^{-K} \cosh(2K) + e^K \cosh(2h) \right] \label{eq:b3_Rec3}
\end{align}
 
Podemos verificar que que se cumplen las relaciones $K'(K, -h) = K'(K, h)$ y $h'(K, -h) = -h'(K, h)$, por lo que $h=0$ es un subespacio invariante. Dado que el punto fijo es no trivial es \'unico, este se encontrar\'a dentro de este subespacio. Para hallarlo, entonces, basta con igualar $h = 0$ en las ecuaciones anteriores, con lo que obtenemos el sistema


\begin{align}
e^{g+K'} &= 2\left[e^{-K} + e^K \cosh(2K) \right] \label{eq:b3_h0_Rec1} \\
e^{g-K'} &= 2\left[e^{-K}\cosh(2K) + e^K \right] \label{eq:b3_h0_Rec2}
\end{align}

Dividiendo  \ref{eq:b3_h0_Rec1} por  \ref{eq:b3_h0_Rec2},

\begin{equation}
e^{2K'} = \dfrac{2e^{-K} + e^K(e^{2K} + e^{-2K})}{e^{-K}(e^{3K} + e^{-2K}) + 2e^K} = \dfrac{e^{2K} + 3e^{K}}{e^{-3K}+3e^{-K}}.
\end{equation}

Luego,

\begin{align*}
\tanh(K') &= \dfrac{e^{2K'}+1}{e^{2K'}-1} \\
&= \dfrac{\dfrac{e^{3K} + 3e^{K}}{e^{-3K}+3e^{-K}} + 1}{\dfrac{e^{3K} + 3e^{K}}{e^{-3K}+3e^{-K}} - 1} \\
&= \dfrac{e^{3K} + 3e^{K} + e^{-3K}+3e^{-K}}{e^{3K} + 3e^{K}-e^{-3K}-3e^{-K}} \\
&= \tanh^3(K)
\end{align*}

Luego, utilizando el mismo cambio de variables que en el problema anterior, tenemos la ecuaci\'on de transformaci\'on 

\begin{equation}
t' = t^3.
\end{equation}

Una vez obtenida la relaci\'on $K'_b(K)$, utilizamos la aproximaci\'on de M-K, cuya expresi\'on para $b = 3$ y $d = 2$ es $K'(K) = 3 K'_3(K)$, es decir,

\begin{equation}
K' = 3 \tanh^{-1}\left[\tanh^3(K)\right].
\end{equation}

La ecuaci\'on anterior se puede escribir como 

\begin{equation}
\tanh\left( \dfrac{K'}{3} \right) = \tanh^3(K).
\end{equation}

Teniendo en cuenta la relaci\'on 

\begin{equation}
\tanh(x+y) = \dfrac{\tanh(x) + \tanh(y)}{1 + \tanh(x) \tanh(y)}
\end{equation}

y definiendo $z = \tanh(x)$,

tenemos

\begin{equation}
\tanh(2x) = \dfrac{2 z}{1+z^2}
\end{equation}

y

\begin{align}
\tanh(3x) &= \dfrac{z + \tanh(2x)}{1 + z \tanh(2x)} \\
&= \dfrac{z (1 + z^2) + 2z}{1 + z^2 + 2z^2} \\
&= \dfrac{z^3 + 3 z}{3 z^2 + 1}.
\end{align}

Luego, teniendo en cuenta que $\tanh(K') = \tanh(3(K'/3))$,

\begin{align}
t' = \tanh(K') =  \dfrac{t^9 + 3 t^3}{3 t^6 + 1}.
\end{align}

Calculamos los puntos fijos mediante $t^*=t=t'$ y definimos $u=t^2$,

\begin{align}
3u^3+1 &= u^4 + 3 u \\
(u^2-1)(u^2-3u+1) &= 0.
\end{align}

Como la variable $u$ est\'a definida en el intervalo $0 \leq u \leq 1$, las \'unicas dos soluciones posibles son

\begin{align*}
u_1 &= 1 \\
u_2 &= \dfrac{3-\sqrt{5}}{2}
\end{align*}

\pagebreak

\textbf{Problema 4: Redes jer\'arquicas}\\

La igualdad entre el hamiltoniano de sitios y el hamiltoniano renormalizado est\'a dada por 

\begin{align*}
\exp \left[ K' S_1 S_2 + \dfrac{h}{2}\left( S_1 + S_2\right) +g \right] = \sum_{\lbrace \sigma \rbrace} \exp &\bigg[ K(S_1 + S_1) (\sigma_1 + \sigma_2) + K \sigma_1 \sigma_2 \\
&+ h (\sigma_1 + \sigma_2) + \dfrac{h}{2} (S_1 + S_2) \bigg]
\end{align*}

Realizando la traza sobre el miembro derecho de la ecuaci\'on,

\begin{align*}
&\sum_{\lbrace \sigma \rbrace} \exp \bigg[ K(S_1 + S_1) (\sigma_1 + \sigma_2) + K \sigma_1 \sigma_2 + h (\sigma_1 + \sigma_2) + \dfrac{h}{2} (S_1 + S_2) \bigg] = \\
e^{\frac{h}{2} (S_1 + S_2)}  &\sum_{\lbrace \sigma \rbrace} \exp \bigg[ K(S_1 + S_1) (\sigma_1 + \sigma_2) + K \sigma_1 \sigma_2 + h (\sigma_1 + \sigma_2) \bigg] = \\
e^{\frac{h}{2} (S_1 + S_2)} &\bigg[ e^{-K(S_1+S_2)+K-2h} + e^{K(S_1+S_2)+K+2h} + 2e^{-K} \bigg] = \\
2 e^{\frac{h}{2} (S_1 + S_2)}  &\bigg[ e^{K} cosh\bigg( K(S_1+S_2) + 2h \bigg) + e^{-K}\bigg]
\end{align*}

Evaluando para cada combinaci\'on de spines $S_1$ y $S_2$, tenemos el sistema de ecuaciones

\begin{align}
e^{g+K'+h'} &= 2e^{h}\left[e^{-K} + e^K\cosh(4K+2h) \right] \label{eq:rombo_Rec1}\\
e^{g+K'-h'} &= 2e^{-h}\left[e^{-K} + e^K\cosh(4K-2h) \right]\label{eq:rombo_Rec2} \\
e^{g-K'} &= 2\left[ e^{-K} + e^K \cosh(2h) \right]. \label{eq:rombo_Rec3}
\end{align}

Podemos verificar que que se cumplen las relaciones $K'(K, -h) = K'(K, h)$ y $h'(K, -h) = -h'(K, h)$, por lo que $h=0$ es un subespacio invariante. Dado que el punto fijo es no trivial es \'unico, este se encontrar\'a dentro de este subespacio. Para hallarlo, entonces, basta con igualar $h = 0$ en las ecuaciones anteriores, con lo que obtenemos el sistema

\begin{align}
e^{g+K'} &= 2\left[e^{-K} + e^K\cosh(4K) \right] \label{eq:rombo_h0_Rec1}\\
e^{g-K'} &= 4\cosh(K). \label{eq:rombo_h0_Rec2}
\end{align}

Dividiendo  \ref{eq:rombo_h0_Rec1} por \ref{eq:rombo_h0_Rec2}, tenemos

\begin{equation} \label{eq:rombo_Rk}
e^{2K'} = \dfrac{e^{-K} + e^K\cosh(4K)}{2\cosh(K)}.
\end{equation}

Para calcular los puntos fijos, hacemos $K'=K=K^*$ y definimos $u=e^{2K^*}$. La ecuaci\'on anterior queda entonces 

\begin{align}
u &= \dfrac{1 + u\dfrac{u^2 + u^{-2}}{2}}{(u+1)} \\
2 u (u+1) &= 2 + u^3 + u^{-1} \\
u^4 - 2 u^3 - 2u^2 + 2u + 1 &= 0 \\
(u-1)(u+1)(u-1-\sqrt{2})(u-1+\sqrt{2}) = 0.
\end{align}

Las \'unicas ra\'ices positivas son $u_1 = 1$ t $u_2 = 1+\sqrt{2}$. La primera corresponde a $K^* = 0$, es decir, $T=\infty$, por lo que corresponde al estado paramagn\'etico. La segunda corresponde a 

\begin{equation}
K^* = \dfrac{\ln(1+\sqrt{2})}{2},
\end{equation}

que coincide con la temperatura cr\'itica para el modelo de Ising en $d=2$ con red cuadrada.

El exponente cr\'itico $\nu$ viene dado por 

\begin{equation}
\nu^{-1} = y_T = \dfrac{\ln\bigg(\dfrac{dK'}{dK}\bigg\rvert_{K^*} \bigg)}{\ln b}.
\end{equation}

Derivando \ref{eq:rombo_Rk} respecto a $K$,

\begin{align}
e^{2K'} \dfrac{dK'}{dK} &= e^{4K} - \cosh(2K) \\
\dfrac{dK'}{dK}\bigg\rvert_{K^*} &= u - \dfrac{1+u^{-2}}{2}\\
\dfrac{dK'}{dK}\bigg\rvert_{K^*} &= 2\sqrt{2}-1.
\end{align}

Con esto se obtiene $\nu \simeq 1.1486$.
 
\end{document}
