\documentclass[10pt]{article}
\usepackage{amsmath} 

\begin{document}

\textbf{Problema 1: Interacciones de corto alcance}
\\

Sistema de tipo Ising con condiciones de borde peri\'odicas e interacciones de corto alcance a varios vecinos

\begin{equation}
\mathcal{H} = - \sum_{i<j\leq N} J(i-j) \sigma_1 \sigma_j -  B \sum_{i=1}^N \sigma_i,
\end{equation}

$J(n) = 0$, $\forall n>n_0$.

El hamiltioniano puede reescribirse en la forma 

\begin{equation}
\mathcal{H} = -\sum_{k=1}^{n_0} J_k \sum_{i=1}^N \sigma_i \sigma_{i+k} - B \sum_{i=1}^N \sigma_i.
\end{equation}

Supongamos que $N = L n_0$. En ese caso, podemos separar nuestra cadena en bloques de spines consecutivos de tama\~no $n_0$ e introducir la siguiente notaci\'on:

\begin{align*}
\left[ \sigma_1, \sigma_2,\dots ,\sigma_{n_0} \right]
\quad &\rightarrow\quad
\left[ s_1^{(1)}, s_2^{(1)},\dots ,s_{n_0}^{(1)} \right]  \\
\left[ \sigma_{n_0+1},\sigma_{n_0+2},\dots ,\sigma_{2n_0} \right]
\quad &\rightarrow\quad
\left[ s_1^{(2)}, s_2^{(2)},\dots ,s_{n_0}^{(2)} \right] \\
&\dots  \\
\left[ \sigma_{(L-1)n_0+1}, \sigma_{(L-1)n_0+1},\dots ,\sigma_{L n_0} \right] 
\quad &\rightarrow\quad
\left[ s_1^{(L)}, s_2^{(L)},\dots ,s_{n_0}^{(L)} \right]
\end{align*}

As\'i, la energ\'ia del sistema viene dada por tres tipos de interacciones diferentes:

\begin{itemize}
\item Interacciones de spines de un bloque con el campo magn\'etico, \\
\item Interacciones de spines dentro de un bloque, \\
\item Interacciones de spines entre bloques consecutivos.
\end{itemize}

Las contribuciones de los primeros dos tipos dentro de un bloque vienen dadas por 

\begin{align*}
X_{c_j} &\equiv X( s_1^{(j)}, s_2^{(j)},\dots ,s_{n_0}^{(j)}) \\
&= -B\sum_{i=1}^{n_0} s_i^{(j)} - \sum_{k=1}^{n_0} J_k \sum_{i=1}^{n_0-k} s_i^{(j)} s_{i+k}^{(j)}.
\end{align*}

Por otro lado, las interacciones entre bloques vienen dadas por 


\begin{align*}
Y_{c_j, c_{j+1}} &\equiv Y( s_1^{(j)}, s_2^{(j)},\dots ,s_{n_0}^{(j)},\dots , s_1^{(j+1)}, s_2^{(j+1)},\dots ,s_{n_0}^{(j+1)}) \\
&= - \sum_{k=1}^{n_0} J_k \sum_{i=1}^{n_0-k} s_{n+1-i}^{(j)} s_{k-i+1}^{(j+1)}.
\end{align*}

Luego, el hamiltoniano del sistema puede escribirse como

\begin{equation}
\mathcal{H} = \sum_{j=1}^{L} \left(X_{c_j} + Y_{c_j, c_{j+1}}\right)
\end{equation}

La funci\'on de partici\'on del sistema ser\'a entonces 

\begin{align}
\mathcal{Z} &= \sum_{\lbrace c\rbrace} \exp \left[ -\beta \sum_{j=1}^{L} \left(X_{c_j} + Y_{c_j, c_{j+1}}\right) \right] \nonumber \\
&=  \sum_{\lbrace c\rbrace} \prod_{j=1}^L \exp \left[ -\beta\left( \dfrac{X_{c_j}}{2} + Y_{c_j, c_{j+1}} + \dfrac{X_{c_{j+1}}}{2}\right) \right]
\end{align}

Definimos la matriz de transferencia 

\begin{equation}
T(c,c') = \exp \left[ -\beta \left(\dfrac{X_{c}}{2} + Y_{c, c'} + \dfrac{X_{c'}}{2}\right) \right],
\end{equation}

entonces

\begin{align}
\mathcal{Z} &= \sum_{c_1} \sum_{c_2}\dots\sum_{c_L} T(c_1,c_2) T(c_2,c_3)\dots T(c_{L-1},c_L) T(c_L,c_1)\nonumber \\
&= \text{Tr} (T^L).
\end{align}

En el l\'imite termodin\'amico, la funci\'on de partici\'on puede aproximarse por 

\begin{equation}
\mathcal{Z} \sim \lambda_1^L, 
\end{equation}

donde $\lambda_1$ es el mayor autovalor de $T$. 



\textbf{Problema 2: Modelo de Ising antiferromagn\'etico unidimensional}
\\

\begin{equation}
\mathcal{H} = J\sum_{i=1}^N \sigma_i \sigma_{i+1} - B \sum_{i=1}^N \sigma_i- B_a \sum_{i=1}^N (-1)^N \sigma_i.
\end{equation}

Simetrizamos el hamiltoniano y separamos las sumas en t\'erminos pares e impares

\begin{align}
\mathcal{H} &= \sum_{i\,\text{par}} \left[J \sigma_i \sigma_{i+1} - \dfrac{B}{2} \left( \sigma_i + \sigma_{i+1}\right) -\dfrac{B_a}{2} \left( \sigma_i - \sigma_{i+1}\right) \right] + \nonumber \\
&+ \sum_{i\,\text{impar}} \left[J \sigma_i \sigma_{i+1} - \dfrac{B}{2} \left( \sigma_i + \sigma_{i+1}\right) -\dfrac{B_a}{2} \left( \sigma_{i+1} - \sigma_i\right) \right].
\end{align}

Definimos la matriz $A$ tal que 

\begin{align*}
(A)_{\sigma\sigma'} = -\beta\exp\left[ J \sigma \sigma' - \dfrac{B}{2} \left( \sigma + \sigma'\right) -\dfrac{B_a}{2} \left( \sigma - \sigma'\right) \right].
\end{align*}

La funci\'on de partici\'on puede escribirse entonces como

\begin{align}
\mathcal{Z} &= \sum_{\lbrace\sigma\rbrace} (A)_{\sigma_1\sigma_2} (A^T)_{\sigma_2\sigma_3}\dots (A)_{\sigma_{N-1}\sigma_N} (A^T)_{\sigma_N\sigma_1} \nonumber \\
&= \text{Tr}\left( U^{N/2} \right),
\end{align}

donde $U=AA^T$. Como $U$ es sim\'etrica, podemos aplicar el teorema de Perron-Frobenius. El sistema, entonces, est\'a caracterizado por el autovalor m\'as grande de $U$, $\lambda_1$.

Calculamos expl\'icitamente $A$ para obtener los autovalores de $U$.

$$A = 
\begin{bmatrix}
e^{-\beta(J+B)} & e^{\beta(J-B_a)}\\
e^{\beta(J+B_a)} & e^{\beta(J+B)}
\end{bmatrix}
$$


$$U = 
\begin{bmatrix}
e^{-\beta(J+B)} & e^{\beta(J-B_a)}\\
e^{\beta(J+B_a)} & e^{\beta(J+B)}
\end{bmatrix}
\times
\begin{bmatrix}
e^{-\beta(J+B)} & e^{\beta(J+B_a)}\\
e^{\beta(J-B_a)} & e^{\beta(J+B)}
\end{bmatrix}
$$

\end{document}